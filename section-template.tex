\documentclass{article}

\usepackage[utf8]{inputenc}

% use this package to get a sense for eventual length (delete later)
\usepackage{lipsum}

\usepackage[
    style=authoryear-ibid,
    backend=biber,
    sorting=ynt
]{biblatex}

\renewcommand{\postnotedelim}{.\space}

\usepackage{lipsum}

% add more space between footnote line and main text
\addtolength{\skip\footins}{1pc plus 2pt}

\usepackage{changepage}
\usepackage{titlesec} % to center the title of the section
\usepackage[OT1]{fontenc}
\usepackage{lmodern}
% \usepackage{footnote}
% ''hang'' option creates hanging indent between number and text within footnote
\usepackage[hang]{footmisc}

\hbadness=99999 % or any number >=10000
% \hbadness variable doesn't affect typography 
% it just tells TeX the threshold for printing
\hfuzz=12pt % suppress overfull hbox messages
\vfuzz=30pt % suppress overfull vbox messages

\usepackage{newunicodechar}
\newunicodechar{ʼ}{'}
\newunicodechar{Ḥ}{\d{H}} % \d is dot under accent
\newunicodechar{ḥ}{\d{h}}

% https://tex.stackexchange.com/a/552852
\DeclareFontSeriesDefault[sf]{bf}{bx}

\usepackage{sansmathfonts}
\renewcommand{\familydefault}{\sfdefault}

\usepackage{textcomp}
\usepackage{setspace}
\usepackage{hyphenat}
\usepackage{csquotes}
\usepackage[titles]{tocloft}

\renewcommand{\contentsname}{\hfill\bfseries\large Section of ``Arablized Native'' monograph \hfill}   
\renewcommand{\cftaftertoctitle}{\hfill}


\usepackage{url}
% \renewcommand{\UrlFont}{\ttfamily\scriptsize}
\tolerance=9000
\usepackage{xurl}
\usepackage[hidelinks]{hyperref}

% no PDF watermark, refer to GDrive doc for more info ...

% add a dot after the section number
\usepackage{titlesec}


\titleformat{\section}[display]{\centering\raggedright\large\bfseries}{\thesection.}{1ex}{}

\renewcommand*{\mkibid}[1]{\mkbibemph{#1}}

% making footnotes endnotes
\usepackage{endnotes}
% \let\footnote=\endnote

\usepackage{fancyhdr, graphicx,lastpage}
\fancypagestyle{plain}{
  \fancyhf{} % clear existing header/footer entries
  \fancyhead[CO]{\footnotesize{\textsc{\textit{DRAFT SECTION IN UNPUBLISHED MANUSCRIPT}\\ NO DISTRIBUTION PERMITTED EXCEPT AS AUTHORIZED BY AUTHOR \\ (FOR MORE INFO, EMAIL AUTHORADDRESS@EMAILACCOUNT.COM)}}}
    \fancyfoot[L]{\footnotesize{\textsc{Interim draft as of:\\January XX, 202X}}}
    \fancyfoot[C]{\footnotesize{--~\thepage~--}}
    \fancyfoot[R]{\footnotesize{\textsc{$\copyright$ \hspace{.02cm} [Author's Name] \\ Certain rights reserved.}}}
}
\setlength{\headheight}{52pt}% ...at least 51.60004pt

\usepackage{textcomp}
\newcommand{\footibid}{\textit{ibid.}}
\addbibresource{section-template.bib} %Imports bibliography file
\widowpenalty10000
\clubpenalty10000

\setlength{\marginparwidth}{6pt}
\setlength{\marginparsep}{6pt}
\setlength{\emergencystretch}{2em}
\DeclareAutoPunct{.}

\DeclareUnicodeCharacter{202F}{\,}
\DeclareUnicodeCharacter{03C3}{\σ}
\DeclareUnicodeCharacter{041F}{\П}
\DeclareUnicodeCharacter{0442}{\т}
\DeclareUnicodeCharacter{043E}{\о}
\DeclareUnicodeCharacter{02BF}{\ʿ}

\setcounter{tocdepth}{1}
\renewcommand{\cftsecleader}{\bfseries\cftdotfill{\cftdotsep}}
\renewcommand{\cftsecaftersnum}{.} % add dot after section number in TOC

\renewcommand{\thesection}{\arabic{section}.}

% NEW
\renewcommand\cftsecfont{\scshape}

% ------------------------

\begin{document}
%% NEW
% \titleformat{\section}
% %  {\normalfont\scshape\normalsize\bfseries}{\thesection.}{10pt}{\large}
%   {\centering\normalfont\scshape\normalsize\bfseries}{\thesection}{10pt}{\large}
% % \maketitle
% \pagestyle{plain}

\titleformat{\section}
 %  {\normalfont\scshape\normalsize\bfseries}{\thesection.}{10pt}{\large}
   {\normalfont\scshape\normalsize\bfseries}{\thesection}{10pt}{\large}
 % \maketitle
% \pagestyle{plain}

% \onehalfspacing
\begin{spacing}{1.35}

%% Section 7
%% ---------------------------
\setcounter{section}{6}  % Set section number to 3
\section{Section Name Goes Here}
\pagestyle{plain}

 This is a paragraph with a few footnotes in it. The first is ``footnote,'' which is simply text inside the brackets of the footnote;\footnote{The first is the footnote but it could be to a citation that is not listed in the .bib file.}, another is a 
``footfullcite,'' meaning this is the first time the cite has appeared;\footfullcite[][p. 303]{Banton-1983} and a third is a ``footcite'' (meaning, it will reflect that the footfullcite has been cited earlier.)\footcite[][p. 194]{Banton-1983} The rest of this is dummy text using the lipsum package, except that here is a dummy excerpt of a source with appropriate indentation. The first footfullcite was to an book; this one is to a review of the book.\footfullcite[][p. 303]{Ellison-1985} And here is a footcite to the same source:\footcite[][p. 304]{Ellison-1985}

\begin{adjustwidth}{.65cm}{.5cm}
  \begin{singlespace}
  ``One of the major inadequacies of the book is, however, that race relations within certain geographical areas and historical periods are over-simplified to the point of distortion. Banton seems out of touch with modern historical scholarship in his acceptance of the magnitude of the differences between slavery in the Northern and Southern continents of America and his views on most aspects of North American slavery and Reconstruction can be dismissed as ill-informed and naive.''
  \end{singlespace}
  \end{adjustwidth}

\noindent
Note also that the number of this section is not set dynamically. Instead, using the ``setcounter'' directive, the last (dummy) section is specified as a certain number so that the following section is that number + 1. Also, this is a new paragraph because there is a line between it and preceding text, but to make it look like a continuation of the prior paragraph (after the indented excerpt), the ``noindent'' directive is used to suppress the indentation on the first line of the paragraph.

Lastly, another fake excerpt is added here to show the effect of the new paragraph (without noindent) before the dummy text:

\begin{adjustwidth}{.65cm}{.5cm}
  \begin{singlespace}
Do you like LaTeX? If so, you may find this template useful for creating a section to a longer document. When you are finished editing the section, you can drop out the front matter and drop it into your longer document, with a table of contents (not shown here) and a bibliography (which is shown here). Happy LaTeXing!
\end{singlespace}
\end{adjustwidth}

\lipsum[1-1][]
  
  \begin{center}
  *\hspace{15pt} *\hspace{15pt} *
\end{center}

\lipsum[1-2][]

\clearpage

\begin{adjustwidth}{.65cm}{.5cm}
  \begin{center}
    \Large\textbf{Bibliography}
  \end{center}
\end{adjustwidth}
\vspace{-0.5cm}

\printbibliography[
  heading=subbibintoc,
  type=article,
  title={Articles},
  notkeyword=exclude
]

\printbibliography[
  heading=subbibintoc,
  type=book, 
  title={Books},
  notkeyword=exclude
]

\end{spacing}
\end{document}